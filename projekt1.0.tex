% Metódy inžinierskej práce

\documentclass[10pt,twoside,slovak,a4paper]{article}

\usepackage[slovak]{babel}
\usepackage[IL2]{fontenc} % lepšia sadzba písmena Ľ než v T1
\usepackage[utf8]{inputenc}
\usepackage{graphicx}
\usepackage{url} % príkaz \url na formátovanie URL
\usepackage{hyperref} % odkazy v texte budú aktívne
\usepackage{array} % pre tabuľky
\usepackage{cite}

\pagestyle{plain}

\title{PlantPal – Inteligentný asistent pre starostlivosť o rastliny}

\author{Alexej Muška,Justin Miklenčič,Jana Michančova\\[2pt]
	{\small Slovenská technická univerzita v Bratislave}\\
	{\small Fakulta informatiky a informačných technológií}\\
	{\small \texttt{xmuska@stuba.sk,xmiklencic@stuba.sk,xmichncova@stuba.sk}}
	}

\date{\small 11. oktober 2025}

\begin{document}

\maketitle

\begin{abstract}
Tento článok predstavuje projekt \textbf{PlantPal}, inteligentnú aplikáciu, ktorá uľahčuje starostlivosť o rastliny prostredníctvom pripomienok, databázy informácií a plánovania pestovateľských úloh. Aplikácia reaguje na rastúci záujem o ekologický životný štýl a domáce pestovanie rastlín a poskytuje praktický nástroj pre pravidelnú starostlivosť, znižuje riziko uhynutia rastlín a podporuje environmentálne povedomie používateľov.
\end{abstract}

\section{Úvod}

V súčasnosti rastie záujem o domácu zeleň a pestovanie rastlín. Rastliny zlepšujú kvalitu života, prispievajú k zníženiu stresu a zlepšujú mikroklímu domácnosti. Mnoho ľudí však naráža na problémy ako nepravidelné polievanie, nevhodné svetelné podmienky alebo zanedbanie sezónnej starostlivosti, čo vedie k oslabeným alebo uhynutým rastlinám.

Projekt PlantPal vznikol s cieľom ponúknuť používateľom jednoduchý a prehľadný nástroj, ktorý im pomôže udržiavať rastliny zdravé a motivovať ich k pravidelnej starostlivosti.

\section{Motivácia projektu}

Motiváciou je zlepšiť dostupnosť informácií o pestovaní a podporiť ekologické povedomie používateľov. Aplikácia má pomáhať pestovateľom, aby ich rastliny rástli zdravo a predišli častým chybám, ako je prelievanie alebo zanedbanie.

\newpage
\section{Cieľ projektu}

Cieľom projektu je vytvoriť aplikáciu, ktorá:
\begin{itemize}
\item poskytne pripomienky polievania, hnojenia a presádzania rastlín,
\item umožní zaznamenávať rast a stav rastlín (denník rastu),
\item obsahuje databázu informácií o rôznych druhoch rastlín,
\item umožní personalizáciu podľa prostredia a podmienok používateľa,
\item bude mať jednoduché a prehľadné používateľské rozhranie.
\end{itemize}

\section{Popis riešenia}

PlantPal funguje ako digitálny „osobný asistent“ pre rastliny. Používateľ pridá svoje rastliny do aplikácie, ku ktorým sa automaticky priradia odporúčania pre starostlivosť. Aplikácia je navrhnutá tak, aby podporovala pravidelnú starostlivosť a pomáhala predchádzať bežným chybám pri pestovaní. V budúcnosti je možné rozšíriť funkcionalitu o rozpoznávanie rastlín podľa fotografie, komunitné zdieľanie skúseností a integráciu s inteligentnými senzormi.

\section{Výhody a nevýhody projektu}


\begin{table}[h!]
\centering
\begin{tabular}{|>{\raggedright}p{7cm}|>{\raggedright\arraybackslash}p{7cm}|}
\hline
\textbf{Výhody} & \textbf{Nevýhody} \\
\hline
Jednoduchosť používania a prehľadné rozhranie & Potreba vytvoriť rozsiahlu databázu rastlín \\
\hline
Podpora pravidelnej starostlivosti a udržania rastlín zdravých & Technická náročnosť implementácie notifikácií a personalizácie \\
\hline
Ekologický a udržateľný charakter projektu & Účinnosť závisí od pravidelného používania používateľom \\
\hline
Možnosť personalizácie podľa druhu rastliny a prostredia & Potrebné riešenie financovania a údržby aplikácie \\
\hline
Potenciál rozšírenia o nové funkcie (AI rozpoznávanie, komunitné funkcie) & Konkurencia na trhu s podobnými aplikáciami \\
\hline
Podpora environmentálneho povedomia a vzťahu používateľa k rastlinám & - \\
\hline
\end{tabular}
\caption{Výhody a nevýhody projektu PlantPal}
\label{tab:vyhody_nevyhody}
\end{table}

\newpage
\section{Zaver}

Projekt PlantPal spája technologickú inovatívnosť s praktickým a environmentálnym prínosom. Aplikácia pomáha používateľom vytvoriť si pravidelný režim starostlivosti o rastliny, podporuje ekologické povedomie a môže byť základom pre ďalší rozvoj funkcií a komunitných možností. Napriek výzvam spojeným s technickou realizáciou a konkurenciou na trhu má projekt vysoký potenciál byť užitočným a spoločensky prínosným nástrojom pre každodenné pestovanie rastlín.

\section{Nejaká časť} \label{nejaka}

Diagram.~\ref{f:rozhod}. 

\begin{figure*}[tbh]
\centering
\includegraphics[scale=1.0]{diagram.png}
\caption{Diagram}
\label{f:rozhod}
\end{figure*}




\bibliography{literatura}
\bibliographystyle{plain}

\end{document}

